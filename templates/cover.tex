% -*-latex-*-
% 
% For questions, comments, concerns or complaints:
% thesis@mit.edu
% 
%
% $Log: cover.tex,v $
% Revision 1.9  2019/08/06 14:18:15  cmalin
% Replaced sample content with non-specific text.
%
% Revision 1.8  2008/05/13 15:02:15  jdreed
% Degree month is June, not May.  Added note about prevdegrees.
% Arthur Smith's title updated
%
% Revision 1.7  2001/02/08 18:53:16  boojum
% changed some \newpages to \cleardoublepages
%
% Revision 1.6  1999/10/21 14:49:31  boojum
% changed comment referring to documentstyle
%
% Revision 1.5  1999/10/21 14:39:04  boojum
% *** empty log message ***
%
% Revision 1.4  1997/04/18  17:54:10  othomas
% added page numbers on abstract and cover, and made 1 abstract
% page the default rather than 2.  (anne hunter tells me this
% is the new institute standard.)
%
% Revision 1.4  1997/04/18  17:54:10  othomas
% added page numbers on abstract and cover, and made 1 abstract
% page the default rather than 2.  (anne hunter tells me this
% is the new institute standard.)
%
% Revision 1.3  93/05/17  17:06:29  starflt
% Added acknowledgements section (suggested by tompalka)
% 
% Revision 1.2  92/04/22  13:13:13  epeisach
% Fixes for 1991 course 6 requirements
% Phrase "and to grant others the right to do so" has been added to 
% permission clause
% Second copy of abstract is not counted as separate pages so numbering works
% out
% 
% Revision 1.1  92/04/22  13:08:20  epeisach

% NOTE:
% These templates make an effort to conform to the MIT Thesis specifications,
% however the specifications can change. We recommend that you verify the
% layout of your title page with your thesis advisor and/or the MIT 
% Libraries before printing your final copy.
\title{Catch me if you can: BERT tries to Syntax.}

\author{H\'ector Javier V\'azquez Mart\'inez}
% If you wish to list your previous degrees on the cover page, use the 
% previous degrees command:
%       \prevdegrees{A.A., Harvard University (1985)}
% You can use the \\ command to list multiple previous degrees
%       \prevdegrees{B.S., University of California (1978) \\
%                    S.M., Massachusetts Institute of Technology (1981)}
\department{Department of Electrical Engineering and Computer Science}

% If the thesis is for two degrees simultaneously, list them both
% separated by \and like this:
% \degree{Doctor of Philosophy \and Master of Science}
\degree{Master of Engineering in Computer Science and Engineering}

% As of the 2007-08 academic year, valid degree months are September, 
% February, or June.  The default is June.
\degreemonth{February}
\degreeyear{2021}
\thesisdate{January 15, 2021}

%% By default, the thesis will be copyrighted to MIT.  If you need to copyright
%% the thesis to yourself, just specify the `vi' documentclass option.  If for
%% some reason you want to exactly specify the copyright notice text, you can
%% use the \copyrightnoticetext command.  
%\copyrightnoticetext{\copyright IBM, 1990.  Do not open till Xmas.}

% If there is more than one supervisor, use the \supervisor command
% once for each.
\supervisor{Robert C. Berwick}{Professor}

% This is the department committee chairman, not the thesis committee
% chairman.  You should replace this with your Department's Committee
% Chairman.
\chairman{Arthur C. Chairman}{Chairman, Department Committee on Graduate Theses}

% Make the titlepage based on the above information.  If you need
% something special and can't use the standard form, you can specify
% the exact text of the titlepage yourself.  Put it in a titlepage
% environment and leave blank lines where you want vertical space.
% The spaces will be adjusted to fill the entire page.  The dotted
% lines for the signatures are made with the \signature command.
\maketitle

% The abstractpage environment sets up everything on the page except
% the text itself.  The title and other header material are put at the
% top of the page, and the supervisors are listed at the bottom.  A
% new page is begun both before and after.  Of course, an abstract may
% be more than one page itself.  If you need more control over the
% format of the page, you can use the abstract environment, which puts
% the word "Abstract" at the beginning and single spaces its text.

%% You can either \input (*not* \include) your abstract file, or you can put
%% the text of the abstract directly between the \begin{abstractpage} and
%% \end{abstractpage} commands.

% First copy: start a new page, and save the page number.
\cleardoublepage
% Uncomment the next line if you do NOT want a page number on your
% abstract and acknowledgments pages.
% \pagestyle{empty}
\setcounter{savepage}{\thepage}
\begin{abstractpage}
% $Log: abstract.tex,v $
% Revision 1.1  93/05/14  14:56:25  starflt
% Initial revision
% 
% Revision 1.1  90/05/04  10:41:01  lwvanels
% Initial revision
% 
%
%% The text of your abstract and nothing else (other than comments) goes here.
%% It will be single-spaced and the rest of the text that is supposed to go on
%% the abstract page will be generated by the abstractpage environment.  This
%% file should be \input (not \include 'd) from cover.tex.

In order to effectively assess Knowledge of Language (KoL) for any statistically-based Language Model (LM), one must develop a test that is first comprehensive in its coverage of linguistic phenomena; second backed by statistically-vetted human judgement data; and third, tests LMs' ability to track human gradient sentence acceptability judgements.  Presently, most studies of KoL on LMs have focused on at most two of these three requirements at a time.  This thesis takes steps toward a test of KoL that meets all three requirements by proposing the LI-Adger dataset: a comprehensive collection of 519 sentence types spanning the field of generative grammar, accompanied by attested and replicable human acceptability judgements for each of the 4177 sentences in the dataset, and complemented by the Acceptability Delta Criterion (ADC), an evaluation metric that enforces the gradience of acceptability by testing whether LMs can track the human data.

To validate this proposal, this thesis conducts a series of case studies with Bidirectional Encoder Representations from Transformers (Devlin et al. 2018).  It first confirms the loss of statistical power caused by treating sentence acceptability as a categorical metric by benchmarking three BERT models fine-tuned using the Corpus of Linguistic Acceptability (CoLA; Warstadt \& Bowman, 2019) on the comprehensive LI-Adger dataset.  We find that although the BERT models achieve approximately 94\% correct classification of the minimal pairs in the dataset, a trigram model trained using the British National Corpus by Sprouse et al. 2018, is able to perform similarly well (75\%).  Adopting the ADC immediately reveals that neither model is able to track the gradience of acceptability across minimal pairs: both BERT and the trigram model only score approximately 30\% of the minimal pairs correctly.  Additionally, we demonstrate how the ADC rewards gradience by benchmarking the default BERT model using \textit{pseudo log-likelihood} (PLL) scores, which raises its score to 38\% correct prediction of all minimal pairs.

This thesis thus identifies the need for an evaluation metric that tests KoL via gradient acceptability over the course of two case studies with BERT and proposes the ADC in response.  We verify the effectiveness of the ADC using the LI-Adger dataset, a representative collection of 4177 sentences forming 2394 unique minimal pairs each backed by replicable and statistically powerful human judgement data.  Taken together, this thesis proposes and provides the three necessary requirements for the comprehensive linguistic analysis and test of the Human KoL exhibited LMs that is currently missing in the field of Computational Linguistics.
\end{abstractpage}

% Additional copy: start a new page, and reset the page number.  This way,
% the second copy of the abstract is not counted as separate pages.
% Uncomment the next 6 lines if you need two copies of the abstract
% page.
% \setcounter{page}{\thesavepage}
% \begin{abstractpage}
% % $Log: abstract.tex,v $
% Revision 1.1  93/05/14  14:56:25  starflt
% Initial revision
% 
% Revision 1.1  90/05/04  10:41:01  lwvanels
% Initial revision
% 
%
%% The text of your abstract and nothing else (other than comments) goes here.
%% It will be single-spaced and the rest of the text that is supposed to go on
%% the abstract page will be generated by the abstractpage environment.  This
%% file should be \input (not \include 'd) from cover.tex.

In order to effectively assess Knowledge of Language (KoL) for any statistically-based Language Model (LM), one must develop a test that is first comprehensive in its coverage of linguistic phenomena; second backed by statistically-vetted human judgement data; and third, tests LMs' ability to track human gradient sentence acceptability judgements.  Presently, most studies of KoL on LMs have focused on at most two of these three requirements at a time.  This thesis takes steps toward a test of KoL that meets all three requirements by proposing the LI-Adger dataset: a comprehensive collection of 519 sentence types spanning the field of generative grammar, accompanied by attested and replicable human acceptability judgements for each of the 4177 sentences in the dataset, and complemented by the Acceptability Delta Criterion (ADC), an evaluation metric that enforces the gradience of acceptability by testing whether LMs can track the human data.

To validate this proposal, this thesis conducts a series of case studies with Bidirectional Encoder Representations from Transformers (Devlin et al. 2018).  It first confirms the loss of statistical power caused by treating sentence acceptability as a categorical metric by benchmarking three BERT models fine-tuned using the Corpus of Linguistic Acceptability (CoLA; Warstadt \& Bowman, 2019) on the comprehensive LI-Adger dataset.  We find that although the BERT models achieve approximately 94\% correct classification of the minimal pairs in the dataset, a trigram model trained using the British National Corpus by Sprouse et al. 2018, is able to perform similarly well (75\%).  Adopting the ADC immediately reveals that neither model is able to track the gradience of acceptability across minimal pairs: both BERT and the trigram model only score approximately 30\% of the minimal pairs correctly.  Additionally, we demonstrate how the ADC rewards gradience by benchmarking the default BERT model using \textit{pseudo log-likelihood} (PLL) scores, which raises its score to 38\% correct prediction of all minimal pairs.

This thesis thus identifies the need for an evaluation metric that tests KoL via gradient acceptability over the course of two case studies with BERT and proposes the ADC in response.  We verify the effectiveness of the ADC using the LI-Adger dataset, a representative collection of 4177 sentences forming 2394 unique minimal pairs each backed by replicable and statistically powerful human judgement data.  Taken together, this thesis proposes and provides the three necessary requirements for the comprehensive linguistic analysis and test of the Human KoL exhibited LMs that is currently missing in the field of Computational Linguistics.
% \end{abstractpage}

\cleardoublepage

\section*{Acknowledgments}

This is the acknowledgements section. You should replace this with your
own acknowledgements.

%%%%%%%%%%%%%%%%%%%%%%%%%%%%%%%%%%%%%%%%%%%%%%%%%%%%%%%%%%%%%%%%%%%%%%
% -*-latex-*-
