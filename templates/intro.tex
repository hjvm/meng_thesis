%% This is an example first chapter.  You should put chapter/appendix that you
%% write into a separate file, and add a line \include{yourfilename} to
%% main.tex, where `yourfilename.tex' is the name of the chapter/appendix file.
%% You can process specific files by typing their names in at the 
%% \files=
%% prompt when you run the file main.tex through LaTeX.
\chapter*{Introduction}

If I may start a thesis document with an anecdote, you will understand the how and why I chose the topic that I did.

If there was a single common factor across my five and a half years of study at the Massachusetts Institute of Technology, it is that Artificial Neural Networks (henceforth referred to as NNs) are the panacea to all our problems, be it in Artificial Intelligence or otherwise.  This is, of course, the idea the overwhelming majority of my courses within the Electrical Engineering and Computer Science department impressed upon me.  I do not believe it far fetched to say that although one may not explicitly say such a thing in the syllabus, the end result is the communication of the same idea: NNs are the panacea.

In this thesis, I demonstrate my mastery of NNs by arguing the contrary: it is not a matter of time until someone solves all the great computational challenges of the century with a Neural Network.  The reason a Neural Network has not developed the same language competence as we humans is not that they are currently not big enough, it is that they constitute an effort akin to barking up the wrong tree.

Enough with the inflammatory claims, on to the thesis itself.

\section{Language Models}
 



\section{Section sample 2}\label{ch1:sec}

Phasellus sed elit vehicula, gravida odio in, vulputate quam. Quisque in elit enim. Vivamus finibus justo elit, sed semper turpis aliquam porttitor. Nulla posuere bibendum nunc sit amet consequat. Vivamus commodo lorem sed metus fermentum rhoncus. Etiam porta sodales purus, vel aliquet lacus facilisis et. Etiam ornare velit non dui auctor fermentum. In elit augue, fringilla at lacinia at, facilisis sit amet lectus. Sed et hendrerit ex. Morbi tristique felis a augue egestas commodo. Nulla porttitor ut urna nec dignissim. Fusce ac pharetra risus, id rhoncus ligula. Pellentesque euismod viverra sem, vel porttitor libero blandit quis. Phasellus orci augue, mattis nec dolor ut, cursus mattis quam. Sed tincidunt eu metus sed pulvinar. Ut a nulla at leo semper accumsan efficitur eget leo.

Sed vel lectus ut dui tempor molestie. Suspendisse blandit sapien posuere quam tempor lobortis. Duis sollicitudin tincidunt dui, at aliquam lorem dictum sit amet. Aenean congue nibh lectus, ut faucibus turpis facilisis quis. Ut aliquet magna at placerat ultricies. Mauris convallis, risus efficitur gravida dapibus, lacus lorem malesuada ligula, eget porta diam felis non turpis. Nulla sed sem finibus, vehicula quam at, vulputate tellus\footnote{Here is a sample footnote referencing figures~\ref{arm:fig1}
and~\ref{arm:fig2}.}  

Quisque elit enim, molestie ac metus ut, condimentum convallis nibh:

\subsection{Subsection sample}

In tempus ex nibh, non eleifend risus iaculis ac. Vestibulum ante ipsum primis in faucibus orci luctus et ultrices posuere cubilia Curae; Nullam in nisi eu arcu laoreet sollicitudin. Mauris consectetur venenatis arcu id finibus. Aenean pellentesque consectetur erat lacinia vulputate. Praesent tempus tempus lorem at dignissim. Proin at odio vitae tortor sollicitudin pretium. Quisque ac purus eu sem rutrum bibendum.

% This is an example of how you would use tgrind to include an example
% of source code; it is commented out in this template since the code
% example file does not exist. To use it, you need to remove the '%' on the
% beginning of the line, and insert your own information in the call.
%
%\tagrind[htbp]{code/pmn.s.tex}{Code sample}{opt:pmn}

Pellentesque ac leo eget lorem vulputate mattis eu a nisl. Duis elit erat, consectetur vulputate ullamcorper a, finibus quis turpis. Vivamus tincidunt dui id purus bibendum malesuada. Fusce accumsan, ipsum quis feugiat sodales, enim est aliquet leo, ut ornare justo mauris quis ex. Sed eros magna, suscipit et blandit non, pretium id felis. Praesent a vehicula tortor. Donec blandit dolor a ipsum sodales, eget aliquet nisl fermentum.

\subsection{Subsection with list}

Ut sollicitudin, lectus eget posuere porttitor, risus dui facilisis risus, a pharetra lacus elit vel eros. Proin fermentum accumsan mauris, quis posuere nisi pharetra scelerisque. 
\begin{enumerate}
  \item Item 1.
  \item Item 2.
  \item Item 3.
\end{enumerate}

Cras nec ullamcorper mauris. Aliquam erat volutpat. Pellentesque habitant morbi tristique senectus et netus et malesuada fames ac turpis egestas. Suspendisse sed dui ac mi auctor scelerisque. Etiam at semper nisi. Cras nec dolor ac purus feugiat auctor. Nunc eget pulvinar massa.

% This is an example of how you would use tgrind to include an example
% of source code; it is commented out in this template since the code
% example file does not exist.  To use it, you need to remove the '%' on the
% beginning of the line, and insert your own information in the call.
%
%\tgrind[htbp]{code/be.s.tex}{Block Exponent}{opt:be}

\section{Section sample 3}

Quisque sed ultrices leo. Donec vestibulum auctor nibh, at faucibus libero mollis in. Quisque massa lorem, feugiat a lectus in, lobortis volutpat lectus. Donec accumsan dui erat, eu tempor tortor facilisis sed. Nulla ullamcorper augue et sapien dapibus, quis bibendum velit porta. Nullam mattis vehicula tortor porttitor porta. Interdum et malesuada fames ac ante ipsum primis in faucibus. Praesent suscipit, lorem vel viverra rhoncus, turpis orci dignissim dui, bibendum pulvinar justo sem vel lorem. Nam porttitor mollis tristique. Aliquam rhoncus magna quis nisl varius mattis. Sed rhoncus, diam in gravida iaculis, mauris tellus imperdiet turpis, at porttitor est leo vel velit. Praesent faucibus ornare sodales. Sed eu lorem purus.  

\subsection{Another subsection sample}

Nullam rhoncus posuere lacus, id volutpat nisi pulvinar viverra. Quisque quis ultricies ante. Duis sollicitudin sapien nec consequat vehicula. Vestibulum convallis erat in arcu aliquam eleifend. Nunc scelerisque lorem non luctus sodales. Curabitur eleifend odio et sagittis semper. Praesent sodales, diam nec vulputate iaculis, neque leo consectetur nunc, a luctus lacus purus et dui. Sed sit amet tortor ullamcorper, malesuada libero quis, imperdiet tortor. Cras tempor blandit massa, sit amet molestie sapien tincidunt quis. Nullam hendrerit venenatis massa, sed lacinia ligula tincidunt vitae.

Nam efficitur et lacus sed eleifend. Aenean quis ipsum eget leo ultrices ornare. Nullam rhoncus ante odio, at dignissim neque posuere eu. Pellentesque sodales tortor est, nec egestas sapien mollis quis. In lectus sapien, pellentesque congue erat consequat, hendrerit aliquet elit. Pellentesque eleifend purus ac diam bibendum, ac auctor ipsum posuere. Cras suscipit leo nec velit fermentum, id varius erat eleifend. Proin sagittis purus id ante lacinia, et congue eros tincidunt. Pellentesque at cursus tellus. Quisque id semper nunc. Quisque viverra a ex at ullamcorper. Morbi mollis erat at ex viverra fringilla. Proin ante dolor, dignissim sodales nisl ac, finibus egestas urna.

Nulla porta urna at pulvinar consectetur. Pellentesque suscipit, neque vitae ultricies rutrum, eros tellus iaculis dui, nec pulvinar justo nibh eu urna. Ut euismod massa nisi, et bibendum risus placerat quis. Integer pretium nulla id risus lobortis laoreet. Aenean quis quam fringilla, elementum odio non, lacinia purus. Vestibulum dui sapien, mollis sit amet massa vel, egestas faucibus velit. Phasellus non justo ut ante vestibulum dictum. Nam in nibh et libero malesuada aliquet. Donec in ex in magna luctus volutpat.

Sed quis dapibus libero. Curabitur id finibus nulla, sed semper felis. Proin dapibus nulla interdum, bibendum tortor et, blandit sapien. Etiam pretium tristique tortor non lacinia. Aliquam dapibus turpis lorem, sit amet porta ex dignissim vitae. In neque felis, sagittis sed ullamcorper lacinia, lobortis ut turpis. Nam quis aliquet justo. Nam eros mi, aliquam vel massa ac, ornare dignissim erat.  This is done by using some combination of
\begin{eqnarray*}
a_i & = & a_j + a_k \\
a_i & = & 2a_j + a_k \\
a_i & = & 4a_j + a_k \\
a_i & = & 8a_j + a_k \\
a_i & = & a_j - a_k \\
a_i & = & a_j \ll m \mbox{shift}
\end{eqnarray*}
instead of the multiplication.  For example, you could use:
\begin{eqnarray*}
r & = & 4s + s\\
r & = & r + r
\end{eqnarray*}
Or by xx:
\begin{eqnarray*}
t & = & 2s + s \\
r & = & 2t + s \\
r & = & 8r + t
\end{eqnarray*}
Cras pharetra ligula nec lectus bibendum, euismod mattis purus cursus. Nullam ut mi molestie purus ultricies lacinia. Phasellus sed orci ac lacus convallis vestibulum. Quisque id nulla ut ipsum finibus vehicula. Curabitur scelerisque erat lobortis, dapibus purus eget, faucibus sapien. Nam enim leo, faucibus id ante sed, fringilla luctus eros. Morbi vulputate, purus at commodo aliquet, turpis dolor sollicitudin libero, id vehicula risus dui sit amet nulla. Sed auctor efficitur urna. Praesent sagittis tellus ac velit vestibulum dignissim. Vivamus justo enim, pellentesque eu posuere id, mattis vitae felis. Aliquam id tincidunt diam. Class aptent taciti sociosqu ad litora torquent per conubia nostra, per inceptos himenaeos. Pellentesque habitant morbi tristique senectus et netus et malesuada fames ac turpis egestas.